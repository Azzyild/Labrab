\documentclass{article}
\usepackage[utf8]{inputenc}
\usepackage[russian]{babel}
\usepackage{amsmath}
\usepackage{graphicx}
\usepackage[colorinlistoftodos]{todonotes}
\usepackage{geometry}
\usepackage{amstext,amsmath,amssymb}           
\usepackage{bm}                 
\usepackage{indentfirst}       
\usepackage{cite}               
\usepackage{multirow}           
\usepackage{array}            
\linespread{1.3}               
\pagestyle{plain}
\usepackage{xcolor}
\usepackage{hyperref}
\geometry{left=3cm,right=1.5cm,top=2cm,bottom=2cm}

\begin{document}
\begin{titlepage}

 \begin{center}
Министерство образования и науки РФ ФГБОУ ВПО\\
Казанский национальный исследовательский технический университет им. А.Н. Туполева\\
Институт технической кибернетики и информатики\\
Кафедра систем информационной безопасности\\
 \end{center}
 \vspace{5cm}
 \begin{center}
  \LARGE \bf{Отчет по лабороторной работе.\\ 
  По дисциплине WEB-технологии}
 \end{center}
 \begin{center}\large
  Тема лабораторной работы: создание блога с "адимнкой".
 \end{center}
 \vspace{4cm}
 \large
  \begin{flushright}
   \textbf{Выполнили:}\\
    студент группы 4305 \\
    Хуснутдинов И.Р.\\
      \textbf{Проверил:}\\
    Хуснуллин Н.Ф.\\
 \end{flushright}
 \vspace{3cm}
 \begin{center}
  Казань 2013
 \end{center}
\end{titlepage}


\setcounter{page}{2}            
\newpage                        % Начинает текст с новой страницы
 \tableofcontents                % Автоматическое создание оглавления по названиям разделов,   подразделов и т.п.

\newpage
 \section{Введение}
  
    В нашем мире современном.\\
    Тристо миллионов сайтов.\\
    Текста на любую тему.\\
    Триллионы где-то байтов.\\
    Сайт создать-у нас задание.\\
    Тристомиллиннопервый\\
    Проявили мы старания\\
    Не жалея сил и нервов.\\
    Сайт отличным получился\\.
    Он красив, функционален.\\
    И ко мне во сне явился.\\
    В стразах классных Эдвард Каллен.\\
    Это к делу отношения\\ 
    Не имеет никакого.\\
    Так что я прошу прощения.\\
    И про сайт продолжу слово.\\
    Если честно, сайт на тройку.\\
    А мне больше и не надо.\\
    Эта троечка в зачетку -\\
    Это лучшая награда.\\
    
\newpage
 \section{Глава 1}

Тему сразу нам не дали.\\
Все не просто в этом мире\\
Нам загадку загадали.\\
Ну а мы ее решили.\\
Нам за это поручили\\
Сделать клевый "блог с админкой".\\
Мы к задаче приступили.\\
С оптимизмом и улыбкой.\\


\newpage
 \section{Глава 2}

Чтобы сайту появится.\\
Нужно несколько шагов нам\\
 Первый шаг здесь-WEB страница.\\
 Шаг второй здесь код програмный.\\
 Третий шаг про базы данных.\\
    Их создание, подключение.\\
    Много сложностей нежданных.\\
    Мы решили без сомнения.\\
    Все шаги преодолели.\\
    Сайт работает как надо.\\
    Показали мы на деле:\\
    Мы способные ребята!\\




 \subsection{Создание WEB страницы}
 Дальше перехожу на прозу.\\
 Для разработки нашего  WEB приложения была выбрана среда ASP.NET.
 И первый этап создания приложения проходил с помощью встроенного конструктора  WEB страниц. Позже было выявлено, что создание страницы "с нуля" является лучим выбором  из-за отсутствия лишнего кода, который мы не используем, но он предлагается по умолчанию.
 Реализованая WEB страница была предоставлена к проверке на I аттестации.

       
 \subsection{Реализация части программирования относящейся к сайту, создание и подключение БД}
Основной частью этого этапа было создание и подключение БД. Нам было предложено создать SQL или NоSQL базу данных. Для нашего проекта была выбранна NQL база данных. Создание и хранение базы данных осуществялось системой Windows Azure SQL Databases.\\
Windows Azure SQL Databases  это облачный сервис от корпорации Microsoft, предоставляющий
возможность хранения и обработки реляционных данных, а также генерации отчетности. Предостав-
ляет функциональность для различных сценариев синхронизации данных (локальная инфраструкту-
ра<=>облако, облако<=>облако). Является частью Windows Azure.

 \subsection{Составление отчета}
  Условием создания отчета было сделать его в формате TeX/LaTeX. И для выполнения поставленной задачи решено было использовать онлайн редактор www.sharelatex.com. Условием выбора данного ресурса- гибкость системы, ее доступность, удобство так же гарантия работы на любой платформе.
 Почему нельзя было воспользоваться обычным текстовым редактором, мне непонятно.

\newpage
 \section{Заключение}
 В ходе выполнения данной лабораторной работы мы научились создавать WEB приложения, работать с базами данных, выкладывать готовое приложение в сеть интернет.\\
 Так же в ходе работы мы ознакомились с многими приложениями для разработчиков:
 \begin{itemize}
\item Azure
\item GitHub
\item Среда ASP.NET
\item LEd
\item Sharelatex
\end{itemize}
Готовый проект был выложен на хостинг MicrosoftAzure.
\href{http://mintminds.azurewebsites.net}{Кликнуть чтобы перейти по ссылке}.

\end{document}